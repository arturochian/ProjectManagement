\documentclass{article}\usepackage[]{graphicx}\usepackage[]{color}
%% maxwidth is the original width if it is less than linewidth
%% otherwise use linewidth (to make sure the graphics do not exceed the margin)
\makeatletter
\def\maxwidth{ %
  \ifdim\Gin@nat@width>\linewidth
    \linewidth
  \else
    \Gin@nat@width
  \fi
}
\makeatother

\definecolor{fgcolor}{rgb}{0.514, 0.58, 0.588}
\newcommand{\hlnum}[1]{\textcolor[rgb]{0.863,0.196,0.184}{#1}}%
\newcommand{\hlstr}[1]{\textcolor[rgb]{0.863,0.196,0.184}{#1}}%
\newcommand{\hlcom}[1]{\textcolor[rgb]{0.345,0.431,0.459}{#1}}%
\newcommand{\hlopt}[1]{\textcolor[rgb]{0.576,0.631,0.631}{#1}}%
\newcommand{\hlstd}[1]{\textcolor[rgb]{0.514,0.58,0.588}{#1}}%
\newcommand{\hlkwa}[1]{\textcolor[rgb]{0.796,0.294,0.086}{#1}}%
\newcommand{\hlkwb}[1]{\textcolor[rgb]{0.522,0.6,0}{#1}}%
\newcommand{\hlkwc}[1]{\textcolor[rgb]{0.796,0.294,0.086}{#1}}%
\newcommand{\hlkwd}[1]{\textcolor[rgb]{0.576,0.631,0.631}{#1}}%

\usepackage{framed}
\makeatletter
\newenvironment{kframe}{%
 \def\at@end@of@kframe{}%
 \ifinner\ifhmode%
  \def\at@end@of@kframe{\end{minipage}}%
  \begin{minipage}{\columnwidth}%
 \fi\fi%
 \def\FrameCommand##1{\hskip\@totalleftmargin \hskip-\fboxsep
 \colorbox{shadecolor}{##1}\hskip-\fboxsep
     % There is no \\@totalrightmargin, so:
     \hskip-\linewidth \hskip-\@totalleftmargin \hskip\columnwidth}%
 \MakeFramed {\advance\hsize-\width
   \@totalleftmargin\z@ \linewidth\hsize
   \@setminipage}}%
 {\par\unskip\endMakeFramed%
 \at@end@of@kframe}
\makeatother

\definecolor{shadecolor}{rgb}{.97, .97, .97}
\definecolor{messagecolor}{rgb}{0, 0, 0}
\definecolor{warningcolor}{rgb}{1, 0, 1}
\definecolor{errorcolor}{rgb}{1, 0, 0}
\newenvironment{knitrout}{}{} % an empty environment to be redefined in TeX

\usepackage{alltt}
\usepackage{geometry}
\geometry{verbose, tmargin=2.5cm, bmargin=2.5cm, lmargin=2.5cm, rmargin=2.5cm}
\IfFileExists{upquote.sty}{\usepackage{upquote}}{}
\begin{document}

\title{Project Management}
\author{Matthew Leonawicz}
\maketitle





\section{Example usage}
This is how \texttt{projman} functions can be used to create and manipulate a project, using the \texttt{projman} project itself as an example.
The code below is not intended to be run directly, but serves as a guide.

\section{R code}

\subsection{Create a project}
Note that I use my own default path for storing a project when creating a new project.
See the \texttt{projman} [default objects](objects.html "default objects") and [creating a new project](func\_new.html "new project") for more details.

\begin{knitrout}
\definecolor{shadecolor}{rgb}{0, 0.169, 0.212}\color{fgcolor}\begin{kframe}
\begin{alltt}
\hlkwd{source}\hlstd{(}\hlstr{"C:/github/ProjectManagement/code/projman.R"}\hlstd{)}  \hlcom{# Eventually load projman package instead}
\hlstd{proj.name} \hlkwb{<-} \hlstr{"ProjectManagement"}  \hlcom{# Project name}
\hlstd{proj.location} \hlkwb{<-} \hlstd{matt.proj.path}  \hlcom{# Use default file location}

\hlstd{docDir} \hlkwb{<-} \hlkwd{c}\hlstd{(}\hlstr{"Rmd/include"}\hlstd{,} \hlstr{"md"}\hlstd{,} \hlstr{"html"}\hlstd{,} \hlstr{"Rnw"}\hlstd{,} \hlstr{"pdf"}\hlstd{,} \hlstr{"timeline"}\hlstd{)}
\hlkwd{newProject}\hlstd{(proj.name, proj.location,} \hlkwc{docs.dirs} \hlstd{= docDir,} \hlkwc{overwrite} \hlstd{= T)}  \hlcom{# create a new project}

\hlstd{rfile.path} \hlkwb{<-} \hlkwd{file.path}\hlstd{(proj.location, proj.name,} \hlstr{"code"}\hlstd{)}  \hlcom{# path to R scripts}
\hlstd{docs.path} \hlkwb{<-} \hlkwd{file.path}\hlstd{(proj.location, proj.name,} \hlstr{"docs"}\hlstd{)}
\hlstd{rmd.path} \hlkwb{<-} \hlkwd{file.path}\hlstd{(docs.path,} \hlstr{"Rmd"}\hlstd{)}

\hlcom{# generate Rmd files from existing R scripts using default yaml front-matter}
\hlkwd{genRmd}\hlstd{(}\hlkwc{path} \hlstd{= rfile.path,} \hlkwc{header} \hlstd{=} \hlkwd{rmdHeader}\hlstd{())}
\end{alltt}
\end{kframe}
\end{knitrout}

\subsection{Update a project}
Functions can be used to create, read, or update. See [Rmd-related functions](func\_rmd.html "Rmd-related functions").

\begin{knitrout}
\definecolor{shadecolor}{rgb}{0, 0.169, 0.212}\color{fgcolor}\begin{kframe}
\begin{alltt}
\hlcom{# update yaml front-matter only}
\hlkwd{genRmd}\hlstd{(}\hlkwc{path} \hlstd{= rfile.path,} \hlkwc{header} \hlstd{=} \hlkwd{rmdHeader}\hlstd{(),} \hlkwc{knitrSetupChunk} \hlstd{=} \hlkwd{rmdknitrSetup}\hlstd{(),}
    \hlkwc{update.header} \hlstd{=} \hlnum{TRUE}\hlstd{)}

\hlcom{# obtain knitr code chunk names in existing R scripts}
\hlkwd{chunkNames}\hlstd{(}\hlkwc{path} \hlstd{=} \hlkwd{file.path}\hlstd{(proj.location, proj.name,} \hlstr{"code"}\hlstd{))}

\hlcom{# append new knitr code chunk names found in existing R scripts to any Rmd}
\hlcom{# files which are outdated}
\hlkwd{chunkNames}\hlstd{(}\hlkwc{path} \hlstd{=} \hlkwd{file.path}\hlstd{(proj.location, proj.name,} \hlstr{"code"}\hlstd{),} \hlkwc{append.new} \hlstd{=} \hlnum{TRUE}\hlstd{)}
\end{alltt}
\end{kframe}
\end{knitrout}

\subsection{Prepare a project website}
With some additional project-specific setup, files can be generated which will assist in creating a project website.
See [website-related functions](func\_website.html "website-related functions").

\begin{knitrout}
\definecolor{shadecolor}{rgb}{0, 0.169, 0.212}\color{fgcolor}\begin{kframe}
\begin{alltt}
\hlcom{# Setup for generating a project website}
\hlstd{index.url} \hlkwb{<-} \hlkwd{file.path}\hlstd{(rmd.path,} \hlstr{"proj_intro.html"}\hlstd{)}  \hlcom{# temporary}
\hlkwd{file.copy}\hlstd{(index.url,} \hlkwd{file.path}\hlstd{(rmd.path,} \hlstr{"index.html"}\hlstd{))}

\hlstd{proj.title} \hlkwb{<-} \hlstr{"Project Management"}
\hlstd{proj.menu} \hlkwb{<-} \hlkwd{c}\hlstd{(}\hlstr{"projman"}\hlstd{,} \hlstr{"R Code"}\hlstd{,} \hlstr{"All Projects"}\hlstd{)}

\hlstd{proj.submenu} \hlkwb{<-} \hlkwd{list}\hlstd{(}\hlkwd{c}\hlstd{(}\hlstr{"About projman"}\hlstd{,} \hlstr{"Introduction"}\hlstd{,} \hlstr{"Related items"}\hlstd{,} \hlstr{"Example usage"}\hlstd{),}
    \hlkwd{c}\hlstd{(}\hlstr{"Default objects"}\hlstd{,} \hlstr{"divider"}\hlstd{,} \hlstr{"Functions"}\hlstd{,} \hlstr{"Start a new project"}\hlstd{,} \hlstr{"Working with Rmd files"}\hlstd{,}
        \hlstr{"Make a project website"}\hlstd{,} \hlstr{"Github user website"}\hlstd{),} \hlkwd{c}\hlstd{(}\hlstr{"Projects diagram"}\hlstd{,}
        \hlstr{"divider"}\hlstd{,} \hlstr{"About"}\hlstd{,} \hlstr{"Other"}\hlstd{))}

\hlstd{proj.files} \hlkwb{<-} \hlkwd{list}\hlstd{(}\hlkwd{c}\hlstd{(}\hlstr{"header"}\hlstd{,} \hlstr{"proj_intro.html"}\hlstd{,} \hlstr{"code_sankey.html"}\hlstd{,} \hlstr{"example.html"}\hlstd{),}
    \hlkwd{c}\hlstd{(}\hlstr{"objects.html"}\hlstd{,} \hlstr{"divider"}\hlstd{,} \hlstr{"header"}\hlstd{,} \hlstr{"func_new.html"}\hlstd{,} \hlstr{"func_rmd.html"}\hlstd{,}
        \hlstr{"func_website.html"}\hlstd{,} \hlstr{"func_user_website.html"}\hlstd{),} \hlkwd{c}\hlstd{(}\hlstr{"proj_sankey.html"}\hlstd{,}
        \hlstr{"divider"}\hlstd{,} \hlstr{"proj_intro.html"}\hlstd{,} \hlstr{"proj_intro.html"}\hlstd{))}

\hlstd{proj.github} \hlkwb{<-} \hlkwd{file.path}\hlstd{(}\hlstr{"https://github.com/leonawicz"}\hlstd{, proj.name)}

\hlcom{# generate navigation bar html file common to all pages}
\hlkwd{genNavbar}\hlstd{(}\hlkwc{htmlfile} \hlstd{=} \hlkwd{file.path}\hlstd{(proj.location, proj.name,} \hlstr{"docs/Rmd/include/navbar.html"}\hlstd{),}
    \hlkwc{title} \hlstd{= proj.title,} \hlkwc{menu} \hlstd{= proj.menu,} \hlkwc{submenus} \hlstd{= proj.submenu,} \hlkwc{files} \hlstd{= proj.files,}
    \hlkwc{title.url} \hlstd{=} \hlstr{"index.html"}\hlstd{,} \hlkwc{home.url} \hlstd{=} \hlstr{"index.html"}\hlstd{,} \hlkwc{site.url} \hlstd{= proj.github,}
    \hlkwc{include.home} \hlstd{=} \hlnum{FALSE}\hlstd{)}

\hlcom{# generate _output.yaml file Note that external libraries are expected,}
\hlcom{# stored in the 'libs' directory below}
\hlstd{yaml.out} \hlkwb{<-} \hlkwd{file.path}\hlstd{(proj.location, proj.name,} \hlstr{"docs/Rmd/_output.yaml"}\hlstd{)}
\hlstd{libs} \hlkwb{<-} \hlstr{"libs"}
\hlstd{common.header} \hlkwb{<-} \hlstr{"include/in_header.html"}
\hlkwd{genOutyaml}\hlstd{(}\hlkwc{file} \hlstd{= yaml.out,} \hlkwc{lib} \hlstd{= libs,} \hlkwc{header} \hlstd{= common.header,} \hlkwc{before_body} \hlstd{=} \hlstr{"include/navbar.html"}\hlstd{)}
\end{alltt}
\end{kframe}
\end{knitrout}

\end{document}
